\documentclass[a4paper]{article}

\begin{document}

\title{Spec1d - Tools to manipulate 1D data sets in MATLAB}

\author{Des McMorrow}

\date{\today}

\maketitle

\section{Introduction}

Requires MATLAB 5.1 or higher

This collection of routines is designed to facilitate 
the manipulation of 1d spectra from the command line
in MATLAB. Spectra can be loaded into separate workspace 
structures and then processed with simple one-line 
commands. Liberal use is made of the
operator overload feature of MATLAB5 so that operators
such '+' are redefined to work on 1d spectra, including
the correct handling of error propogation.

\subsection{Spec1d Class Structure}

Spec1d objects have the following structure:

x-data    :     s.x
y-data    :     s.y
error in y:     s.e
x-label   :     s.x_label
y-label   :     s.y_label
datafile  :     s.datafile
y-fit     :     s.yfit

\subsection{Command summary}

For the following examples sn (n=1,2...) is a spec1d data type, 
defined in spec1d.m, xc and yc are constants to be applied 
to the x and y axes respectively. 

Command		Description			Examples
---------------------------------------------------------------
+		plus				r=s1+s2+s3+s4;
                                                r=s1+yc;
                                                r=s1+[xc,yc];

-		minus				r=s1-s2;
                                                r=s1-yc;
                                                r=s1-[xc,yc];

.*		multiply			r=s1.*s2; 
                                                r=yc.*s1;
                                                r=[xc,yc].*s1;

./		divide				r=s1./s2;
                                                r=s1./yc;
                                                r=s1./[xc,yc];

[]		append (x values differ)	r=[s1;s2;s3];

combine		combine (x values differ)	r=combine(0.01,s1,s2,s3);

cut             cut data specified by x range   r=cut(s1,[xleft1 xright1],[xleft2 xright2])

disp		display data columns		disp(s1)

extract		extract data from spectra	[x,y,e,yf]=extract(s1); 

fits            fit data to mfit function       [r,f]=fits(s1,'gauss',[100 0.01 0.1 1]);

gfields         get data stored in spectra      [x,y,e,x_lab]=gfields(s1,'x','y','e','x_label')

loads		load data into spec1d object	see below

peakm           calculate peak parameters       r=peakm(s1,[bgleft bgright])
                (moment method)

peakt           calculate integrated intensity  r=peakt(s1)
                (trapezoidal method)

plot		simple plot of spectra		plot(s1,s2,s3,'semilogy')

rebin

sfields         set fields in spectra           sfields(s1,'x',[1 2 3 4],'y_label','Intensity')

transform       transform data                  r=transform('normalise',s1,1);
                                                r=transform('dydx',s1);


\section{Reference}

\subsection{Loading spectra}

Spectra can be loaded using the command file loads.
This makes use of the standard MFIT/MVIEW load routines
to load the data into a data structure.

For example, the SPEC data file 'spec.01' maybe loaded into
the data structure 's1' using the following command

>>s1=loads('specbatch','spec.01,X=K,Y=Detector,S=3');

Here 'specbatch' specifies the standard
MFIT/MVIEW load routine for SPEC type data files, and
the line 'spec.01,X=K,Y=Detector,S=3' causes the
3rd scan from the file 'spec.01' to be loaded with 
the x varible equal to 'K' and the Y variable set to
'Detector'.

s1 is a data structure defined in 'spec1d.m' to be

s1.x		:  x-data
s1.y		:  y-data
s1.e		:  error on y
s1.x_label	: Label for x-axis
s1.y_label	: Label for y-axis
s1.datafile	: Data file from which the scan was loaded
s1.yfit         : Fitted value of data

\subsection{Basic plotting}

A basic plot of the spectra can be generated with the
plot command, eg.

>>plot(s1);

Two or more spectra can be overlayed

>>plot(s1,s2,s3);

Most properties of the graph can be edited by clicking on the 
appropriate feature. Left click on the lables to edit them.
Right click on the lines to edit them.

\subsection{Simple operations}

Simple arithmetic operations may be performed on the 
spectra using the plus '+', minus '-', element-by-element
multiplication '.*', and simple element-by-element divide
'./'. 

Note: these operations require that the spectra have 
identical x-values. If this is not the case then the
spectra have to be first interpolated, or in the case of
'+' 'combine' may be used.

For example, suppose we wish to load scans 3, 4 and 5 from
file 'spec.01' and then process them using simple 
arithmetic.

First load the scans into workspaces:

>>s3=loads('specbatch','spec.01,X=K,Y=Detector,S=3');
>>s4=loads('specbatch','spec.01,X=K,Y=Detector,S=4');
>>s5=loads('specbatch','spec.01,X=K,Y=Detector,S=5');

We can now add the three spectra and view the result 
by typing:

>>r=s3+s4+s5;
>>plot(r);

Alternatively s5 could be subtracted from the sum of 
s3 and s4.

>>r=s3+s4-s5;
>>plot(r);

The operators '.*' and './' work in a similar way,
e.g.

>>r=s3.*s4;

or even

>>r=2.*s4-s3;

It is possible to append spectra as follows

>>r=[s1;s2;s3;s4];

\subsection{Other operations}

These will be added to with time.
See the help section of the individual commands. 
Presently supported commands include:

\subsubsection{combine}

This is used to combine data points when averaging spectra that 
do not have the same x-values, e.g.,

>>r=combine(0.01,s1,s2,s3);

Averages points in spectra s1, s2, and s3 that do not differ by 
0.01 in their x values. This can also be used as a crude rebinning
of one spectra

>>r=combine(0.01,s1);

\subsubsection{cut}

\subsubsection{disp}

\subsubsection{extract}

\subsubsection{fits}

\subsection{formatpars}

\subsection{gfields}

\subsubsection{peakm}

\subsubsection{peakt}

\subsection{rebin}

\subsection{sfields}

\subsubsection{transform}

spectrum_out=transform(transform_function,spectrum_in,args)

Transform spectrum according to the transform function, e.g.,

>>r=transform('normalise',spectra,1)
 
Currently available transform functions include:

- r=transform('normalise',spectra,normalisation_value);
- r=transform('dydx'     ,spectra);


\section{Installation}

- Requires ver 5.0 of MATLAB or later.
- Download latest version of load routines from:

http://www.ill.fr/tas/matlab/ftp/load.tar.gz

-You should create a directory called 'spec1d' in a directory which
should be entered in your matlab path. Below this directory create
another directory called @spec1d. Place the files formatpars.m,
loads.m, specdfdp.m, and speclsqr.m in the spec1d directory, and
all other files in @spec1d.

Obtaining latest Matlab stuff (MFit/MView/Rescal/SSym/load/funcs/nllsq)
is easy on URL :

http://www.ill.fr/tas/matlab/doc/matprgs.html

\section{In the pipeline}

\end{document}




